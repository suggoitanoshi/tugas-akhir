\documentclass[aspectratio=169]{beamer}
\usefonttheme[onlymath]{serif}

\usepackage[utf8]{inputenc}

\usepackage[style=apa]{biblatex}
\usepackage{mathtools}
\renewcommand*{\nameyeardelim}{\addcomma\addspace}
\addbibresource{references.bib}

\title[Modifikasi Adam]{Pengembangan dan Implementasi Modifikasi Pengoptimasi Adam pada Lingkungan Terdistribusi}
\author[]{
  Fransiskus Febryan Suryawan\\
  13519124
}
\institute{Institut Teknologi Bandung}
\date{}

\usetheme{venti}

\begin{document}

{
\makeatletter
\setlength{\hoffset}{-.5\beamer@sidebarwidth}
\makeatother
\begin{frame}[plain]
  \titlepage
\end{frame}
}

\section*{Outline}
\begin{frame}{Outline}
  \begin{center}
    \tableofcontents
  \end{center}
\end{frame}

\section{Pendahuluan}
\subsection{Latar Belakang}
\begin{frame}{Deep Learning}
  \begin{itemize}
    \item Deep Learning banyak digunakan dalam banyak domain
    \item Sebuah model Deep Learning memiliki parameter yang dapat dipelajari
    \item Pembelajaran pada Deep Learning banyak menggunakan \textit{backpropagation} untuk pembelajaran
    \item Pengoptimasi untuk pembelajaran: SGD, \textbf{Adam}, RMSProp, ...
    \item Arah perkembangan model Deep Learning umumnya adalah memperbanyak parameter untuk mendapat hasil yang lebih baik
  \end{itemize}
\end{frame}

\begin{frame}{Distributed Deep Learning}
  \begin{itemize}
    \item Beberapa mesin dapat digunakan secara bersamaan untuk mempercepat pembelajaran model Deep Learning
    \item \textit{Bottleneck}: adanya \textit{straggler}, \textit{bandwidth}, ...
    \item Terdapat banyak penelitian untuk mengurangi efek dari \textit{straggler}
    \item Terdapat usaha untuk mengurangi kebutuhan \textit{bandwidth} untuk bertukar parameter selama pembelajaran
          \begin{itemize}
            \item Kompresi: \textit{lossless}, \textit{lossy}
            \item Pengurangan komunikasi
          \end{itemize}
  \end{itemize}
\end{frame}

\begin{frame}{Pengoptimasi Adam}
  \begin{enumerate}
    \item Pengoptimasi digunakan untuk mempelajari parameter pada model
    \item Adam \parencite{ADAMKingma} menggunakan \textit{learning rate} adaptif untuk setiap parameter
    \item Adam menggunakan estimasi momen orde pertama dan kedua untuk membantu pembelajaran
    \item Terdapat modifikasi Adam untuk pembelajaran model Deep Learning Terdistribusi
          \begin{itemize}
            \item CADA: Communication Adaptive Distributed Adam - pengurangan komunikasi \parencite{Chen2021CADA}
            \item Efficient-Adam: kompresi bobot menggunakan kuantisasi \parencite{Chen2022Efficient}
          \end{itemize}
    \item Belum ada modifikasi Adam yang mengurangi komunikasi serta melakukan kuantisasi secara bersamaan
  \end{enumerate}
\end{frame}

\subsection{Rumusan Masalah}
\begin{frame}{Tujuan}
  \begin{itemize}
    \item Menggabungkan teknik pada CADA dengan Efficient-Adam
    \item Membandingkan hasil modifikasi yang telah dibuat dengan CADA
    \item Membandingkan hasil modifikasi yang dibuat dengan Efficient-Adam
  \end{itemize}
\end{frame}

\section{Penelitian Terkait}
\subsection{CADA}
\begin{frame}{CADA}
  \begin{itemize}
    \item Terdapat teknik \textit{Lazily Aggregated Gradient} \parencite{Chen2018LAG}
          \begin{itemize}
            \item Keefektifannya berkurang seiring berjalannya pembelajaran
          \end{itemize}
    \item Pengurangan komunikasi berdasarkan kriteria tertentu
    \item Memilih \textit{worker} yang akan memperbarui parameter
          \begin{itemize}
            \item CADA 1: Membandingkan perkembangan gradien sejak \textit{update} terakhir
                  \begin{equation*}
                    \| \delta_m^k - \tilde{\delta}_m^{k-\tau_m^k} \| \le \frac{c}{d_{\mathrm{max}}}\sum_{d=1}^{d_{\mathrm{max}}}\| \theta^{k+1-d} - \theta^{k-d} \|^2
                  \end{equation*}
            \item CADA 2: Membandingkan gradien hasil pembaruan parameter sejak \textit{update} terakhir
                  \begin{equation*}
                    \| \nabla \ell (\theta^k; \xi_m^k) - \nabla \ell (\theta_m^{k-\tau_m^k}; \xi_m^k) \|^2 \le \frac{c}{d_{\mathrm{max}}} \sum_{d=1}^{d_{\mathrm{max}}} \| \theta^{k+1-d} - \theta^{k-d} \|^2
                  \end{equation*}
          \end{itemize}
  \end{itemize}
\end{frame}

\subsection{Efficient-Adam}
\begin{frame}{Efficient-Adam}
  \begin{itemize}
    \item Kuantisasi untuk mengurangi ukuran parameter serta gradien
          \begin{itemize}
            \item Mengakibatkan presisi turun, mungkin terjadi bias
          \end{itemize}
    \item Error-feedback digunakan untuk mengurangi efek bias
  \end{itemize}
\end{frame}

\section{Rancangan Solusi}
\begin{frame}{Rancangan Solusi}
  \begin{itemize}
    \item Menggunakan pustaka \texttt{TensorFlow} untuk abstraksi model Parameter Server serta pembangunan model Deep Learning
    \item Mendesain modifikasi Adam yang menggabungkan CADA dan Efficient-Adam
          \begin{itemize}
            \item Harapan: Mengurangi kebutuhan \textit{bandwidth} lebih jauh
          \end{itemize}
  \end{itemize}
\end{frame}

\end{document}
