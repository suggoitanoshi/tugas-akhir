\chapter{Rencana Penyelesaian Masalah}

\section{Analisis Masalah}

Pelaksanaan pembelajaran pada lingkungan terdistribusi membutuhkan komunikasi untuk melakukan sinkronisasi bobot. Kebutuhan sinkronisasi dapat menjadi penghambat pembelajaran karena perlunya saling menunggu. Selain itu, adanya batasan \emph{bandwidth} juga mengharuskan pengurangan komunikasi. Oleh karena itu, penelitian dari \textcite{Chen2021CADA} mencoba untuk mengurangi jumlah komunikasi yang digunakan dalam melakukan sinkronisasi. Selain itu, terdapat penelitian dari \textcite{Chen2022Efficient} yang mengurangi \emph{bandwidth} yang diperlukan untuk sinkronisasi bobot. Keduanya menggunakan pendekatan yang berbeda untuk mengurangi hambatan pembelajaran yang muncul akibat komunikasi yang diperlukan. Namun, belum ada modifikasi untuk pengoptimasi Adam yang menggunakan kedua konsep tersebut untuk lebih jauh mengurangi hambatan komunikasi.

Konsep pengurangan komunikasi dan kuantisasi dapat digabungkan menjadi satu algoritma pengoptimasi yang lebih efisien. Namun, hasil penggabungan keduanya tidak dapat dipastikan konvergen. Selain itu, perlu untuk membandingkan penggunaan \emph{bandwidth} antara algoritma hasil perancangan dengan algoritma dari \textcite{Chen2021CADA} dan \textcite{Chen2022Efficient}. Penggabungan kedua algoritma juga tidak dapat dilakukan secara langsung, karena ada perbedaan cara pembaruan parameter. Dalam algoritma dari \textcite{Chen2021CADA}, pembaruan parameter dilakukan pada server dan seluruh parameter disebarkan kembali kepada seluru \emph{worker}. Di sisi lain, algoritma dari \textcite{Chen2022Efficient} hanya mempertukarkan gradien yang dimiliki, dengan menambahkan \emph{error-feedback} dalam setiap iterasi. Oleh karena itu, perlu adanya modifikasi agar keduanya dapat digabungkan.

\section{Analisis Solusi}
Untuk menyelesaikan masalah yang telah dituliskan pada subbab sebelumnya, maka akan dirancang suatu algoritma untuk mengoptimasi penggunaan bandwidth dan komunikasi bagi pengoptimasi Adam pada lingkungan terdistribusi. Optimasi komunikasi dan penggunaan bandwidth dapat dilakukan dengan menggunakan teknik yang digunakan pada \textcite{Chen2021CADA} dan \textcite{Chen2022Efficient} secara berurutan.

Pengoptimasi yang dirancang akan didasarkan pada pengoptimasi Adam. Pengoptimasi Adam dipilih sebagai dasar karena pengoptimasi tersebut merupakan pilihan yang umum digunakan dalam berbagai kasus. Selanjutnya, kuantisasi akan dilakukan dengan teknik yang dijelaskan oleh \textcite{Chen2021CADA}. Setelah kuantisasi dilakukan pada parameter, pengurangan komunikasi dapat dilakukan dengan menggunakan teknik yang dijelaskan oleh \textcite{Chen2022Efficient}. Kedua optimasi dilakukan untuk mengurangi lebih jauh batasan kecepatan pembelajaran yang muncul akibat komunikasi.

\section{Rancangan Penyelesaian Masalah}
Rancangan solusi akan dibuat dengan menggunakan bantuan pustaka PyTorch. Pemilihan PyTorch sebagai pustaka utama didasarkan kepada adanya referensi implementasi dari penelitian \textcite{Chen2021CADA} menggunakan PyTorch. Selain itu, PyTorch juga memiliki \emph{application programming interface (API)} untuk menulis program terdistribusi, termasuk \emph{deep learning} terdistribusi.

Implementasi dilakukan dengan menggunakan referensi sumber kode algoritma \textcite{Chen2021CADA}\footnote{\url{https://github.com/ChrisYZZ/CADA-master}}. Kemudian, kode tersebut dimodifikasi sehingga dapat diturunkan menjadi implementasi yang berbeda-beda. Implementasi CADA kemudian dipindahkan menjadi turunan dari kode dasar tersebut. Selain itu, Efficient-Adam juga harus diimplementasikan karena tidak ada implementasi yang dapat dijadikan referensi. Oleh karena itu, Efficient-Adam diimplementasikan menjadi turunan kode dasar yang telah dibuat.

Penggabungan CADA dengan Efficient-Adam dilakukan dengan memodifikasi implementasi CADA pada bagian pembaruan parameter serta perhitungan gradien. Karena adanya perubahan perhitungan gradien, maka perlu diubah pula kondisi yang digunakan dalam CADA agar menggunakan perhitungan gradien yang baru. Teknik gabungan akan memiliki \emph{hyperparameter} yang sama dengan CADA, karena didasarkan pada implementasi CADA.

Untuk pemilihan pemetaan kuantisasi, dibutuhkan sebuah pemetaan kuantisasi yang memenuhi \autoref{efficientquant}. Beberapa contoh pemetaan kuantisasi diberikan dalam penelitian \textcite{Chen2022Efficient}. Untuk pengujian, pemetaan yang digunakan adalah pemetaan dari tipe data \emph{float64} ke tipe data \emph{float16}. Pemetaan ini dipilih karena memberikan rasio kompresi yakni 4 kali, serta implementasinya yang sederhana dalam PyTorch. Selain itu, pemetaan tersebut juga memenuhi \autoref{efficientquant}, dengan memilih $\delta = 1$ dan $\delta' = \|x - \mathcal{Q}(x)\|$.
