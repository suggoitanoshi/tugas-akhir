\subsection{Analisis Hasil Pengujian}
\begin{frame}{Analisis Hasil Pengujian}
  \begin{itemize}
    \item Teknik gabungan menggunaka jumlah byte yang lebih sedikit dibanding CADA, namun lebih banyak dibandingkan Efficent-Adam.
    \item Jumlah byte tambahan yang digunakan dalam teknik gabungan adalah komunikasi nilai ambang.
    \item Pengurangan jumlah komunikasi dalam skenario Fashion-MNIST tidak ada, karena jumlah parameter yang lebih sedikit, nilai ambang akan lebih kecil sehingga lebih mudah untuk mencapai nilai tersebut.
    \item Pengurangan jumlah komunikasi dalam skenario CIFAR10 lebih banyak, karena nilai ambang yang lebih besar.
    \item Jumlah komunikasi pada CADA tidak berkurang banyak karena pemilihan \textit{hyperparameter} yang menyebabkan selisih gradien besar.
    \item Pemilihan \textit{hyperparameter} yang berbeda menyebabkan model lebih sulit konvergen ke akurasi yang lebih tinggi.
  \end{itemize}
\end{frame}
